\documentclass[10pt,a4paper,oneside]{scrbook}
\usepackage[utf8]{inputenc}
\usepackage[italian]{babel}
\usepackage{amsmath}
\usepackage{amsfonts}
\usepackage{amssymb}
\usepackage[dvipsnames]{xcolor}
\usepackage{graphicx}
\usepackage{frontespizio}
\usepackage{microtype}
\usepackage{textcomp}
\usepackage{hyperref}
\hypersetup{
    colorlinks,
    citecolor=black,
    filecolor=black,
    linkcolor=black,
    urlcolor=black
}
\usepackage{listings}
\usepackage{tikz}
\usepackage{pgfplots}
\usepackage{enumitem}
\usetikzlibrary{tikzmark}
\usetikzlibrary{calc}
\usetikzlibrary{shapes,snakes}
\usepackage{minted}

\begin{document}
\begin{frontespizio}
    \Universita{Verona}
    \Dipartimento{Informatica}
    \Corso{Ingegneria Informatica}
    \Annoaccademico{2018--2019}
    \NCandidati{Autori}
    \Candidato{Matteo Iervasi}
    \Candidato{Linda Sacchetto}
    \Candidato{Leonardo Testolin}
    \Titolo{Appunti di Progettazione di sistemi embedded}
\end{frontespizio}

\tableofcontents
\newpage

\chapter*{Prefazione}
Questa dispensa si basa sugli appunti di Linda Sacchetto e Leonardo Testolin durante il
corso di \textit{Progettazione di Sistemi Embedded} dell'anno accademico 2018/2019.
Nonostante sia stata revisionata in corso di scrittura, potrebbe contenere errori di vario tipo.
In tal caso potete segnalarli inviando una mail all'indirizzo \href{mailto:matteoiervasi@gmail.com}{matteoiervasi@gmail.com}.

\hspace*{\fill} Matteo Iervasi

\chapter{Introduzione}
Citando Wikipedia, un sistema embedded, nell'informatica e nell'elettronica,
identifica genericamente tutti quei sistemi elettronici di elaborazione digitale a microprocessore progettati
appositamente per una determinata applicazione (special purpose), ovvero non riprogrammabili dall'utente per altri scopi,
spesso con una piattaforma hardware ad hoc, integrati nel sistema che controllano ed in grado di gestirne tutte o
parte delle funzionalità richieste.

Storicamente, sono nati prima i sistemi embedded rispetto ai sistemi \textit{general purpose}, basti pensare ai
grandi calcolatori degli anni '40. Essi infatti erano costruiti per un utilizzo specifico, anche se in quanto a 
dimensioni non erano di certo ristretti.
Tuttavia il primo vero sistema embedded, in tutti i sensi, fu l'\textit{Apollo Guidance Computer}, che doveva
contenere una notevole potenza computazionale per il tempo in spazi ristrettissimi. La produzione di massa di 
sistemi embedded cominciò con l'\textit{Autonetics~D-17} nel 1961 e continua fino ai nostri giorni.

Non possiamo progettare i sistemi embedded come facciamo con i sistemi \textit{general purpose}, perché abbiamo
dei vincoli di progettazione e degli obiettivi differenti. Se ad esempio nei sistemi \textit{general purpose} la
ricerca si focalizza nel costruire processori sempre più veloci, nei sistemi embedded la CPU esiste solamente come
un modo per implementare algoritmi di controllo che comunica con sensori ed attuatori, e diventa invece più 
interessante trovare processori che usano sempre meno energia.

I vincoli principali ai quali bisogna attenersi durante la progettazione di un sistema embedded sono:
\begin{itemize}
    \item \textbf{Dimensione e peso}\\
          Si pensi ai dispositivi che devono poter essere tenuti in una mano
    \item \textbf{Energia}\\
          Molto spesso i dispositivi embedded devono funzionare con una batteria
    \item \textbf{Ambiente esterno ostile}\\
          Bisogna dover tenere conto di eventuali fluttuazioni di energia, interferenze radio, calore, acqua, ecc.
    \item \textbf{Sicurezza e operazioni \textit{real time}}\\
          Vi sono casi in cui è necessario che il sistema debba garantire sempre il funzionamento, oppure che 
          garantisca un tempo costante per ogni operazione
    \item \textbf{Costi contenuti}\\
          Oltre a tutto il resto, bisogna tenere i costi bassi altrimenti si rischia di non poter vendere il prodotto
\end{itemize}

\chapter{Modellazione dei sistemi embedded}
Nel momento in cui ci accingiamo a pensare a come si progetta un sistema embedded, salta subito alla mente un
problema, ovvero come faccio a verificarne il corretto funzionamento?
Quando progettiamo del software, abbiamo a disposizione una miriade di strumenti per assicurarci di tenere il numero
dei bug il più basso possibile: \textit{debugger}, \textit{unit testing}, \textit{analisi statica}, ecc.
In hardware invece non possiamo certamente metterci a rifare tutto ogni volta che sbagliamo, ricordiamoci che dobbiamo
tenere i costi bassi! Si pone quindi il problema della \textit{simulazione}, strumento fondamentale per la verifica
del nostro sistema. Spesso infatti l'architettura di riferimento è differente da quella del calcolatore che usiamo
per lo sviluppo (caso tipico: noi sviluppiamo su architettura X86 per un'architettura di destinazione ARM).

\smallskip
In generale, un sistema embedded è costituito dalle seguenti componenti:
\begin{itemize}
	\item \textbf{Piattaforma hardware}\\
    Oltre al microprocessore, vi sono una serie di altre componenti.
	\item \textbf{Componenti software}\\
    Il software è monolitico: quando accendo il sistema, si esegue in automatico.
    Possono anche esserci casi in cui è necessario caricare un intero sistema operativo, e nella maggioranza di essi si fa riferimento a Linux.
	\item \textbf{Componenti analogiche} (es. sensori e trasduttori)\\
    Naturalmente se il nostro sistema dovrà interagire con l'ambiente esterno dovremmo introdurre componenti analogiche.
\end{itemize}
Il trend attuale è di portare tutto ciò in un singolo chip (SoC - \textit{System on a Chip}), dove microprocessore,
memoria e altre componenti vengono montate su un singolo chip, collegate da un bus. Esempi di SoC moderni sono i
Qualcomm\textsuperscript{\tiny\textregistered} Snapdragon o i Samsung\textsuperscript{\tiny\textregistered} Exynos.
Esiste anche un'alternativa, dove le componenti invece di trovarsi in un unico chip si trovano in una singola
board (SoB - \textit{System on a Board}).

Queste due tecnologie hanno uno scopo in comune: indurre al riutilizzo di componenti già esistenti per
ridurre il più possibile il \textit{time to market} (figura \ref{img:time-to-market}), ovvero il tempo che trascorre
dall'inizio della progettazione all'immissione nel mercato. Più questo tempo è lungo, meno probabilità si ha di riuscire a vendere
il prodotto in quantità tale da coprire i costi di sviluppo.
Per questo motivo non si può minimamente pensare di sviluppare un sistema embedded partendo da zero, in quanto il tempo di sviluppo
sarebbe improponibile.

\begin{figure}
	\centering
	\caption{Rappresentazione del rapporto costo produzione - guadagno}
	\label{img:time-to-market}
	\begin{tikzpicture}
		\centering
		\begin{axis}[
			axis lines=left,
			width=\textwidth,
			height=200pt,
            xmin=0,
			xlabel=Tempo,
			ylabel=Costo,
            yticklabels={,,},
            xticklabels={,,},
		]
		\addplot[
			domain=0:7,
			samples=150,
			color=blue,
			line width=1.25pt,
		]{5*ln(x-1/2)+3};
        \addlegendentry{Costo di produzione}
        \addplot[
            domain=1.25:7,
   			samples=150,
   			color=red,
   			line width=1.25pt,
        ]{20/ln(x)-20};
        \addlegendentry{Prezzo di vendita}
		\end{axis}
        \draw [dashed] (0.9,0) node [below]{$t_0$};
        \draw [dashed] (2,0) node [below]{$t_m$} -- (2,5.5);
        \draw [dashed] (3.6,0) -- (3.6,1.4);
        \draw [<->](2,0.70) -- (3.6,0.70) node [below, midway] {$\Delta_t$};
        \draw [->] (0.9,-0.45) -- (1,-1) node [below] {Inizio realizzazione oggetto};
        \draw [->] (2,-0.45) -- (4,-1.5) node [below] {Inizio vendita oggetto};
	\end{tikzpicture}
\end{figure}

\section{Co-design di sistemi embedded}
Viste le ristrettezze imposte sui tempi dal mercato, è necessario ricorrere al co-design di hardware e software.
Si parte quindi con una descrizione generale del sistema, magari aiutandosi con un prototipo, dopodiché fatte le verifiche su di esso si procede con la separazione di
hardware e software, ricordandosi che è necessario ‘‘riciclare’’ il più possibile le parti già esistenti, si comincia lo sviluppo o l'eventuale adattamento.
Durante la progettazione della parte hardware, si può decidere di affiancare l'eventuale processore da un co-processore, che svolge un compito
specifico a seconda di cosa stiamo progettando. Il co-processore può essere anche una GPU.

Le tecnologie più usate per la progettazione di hardware sono:
\begin{itemize}
	\item Microcontrollore standard o microprocessore
	\item ASIC (con o senza co-processore a seconda dei casi)
	\item FPGA
\end{itemize}
mentre per il software si possono utilizzare diversi linguaggi di programmazione (generalmente però si scelgono C/C++).

Naturalmente ci serve uno strumento per la verifica del nostro sistema, ci serve quindi co-simulare hardware e software.
Si può fare su diversi livelli, ognuno con i suoi pregi e i suoi difetti.
\begin{itemize}
	\item \textbf{Gate level}\\
	Viene simulato il tutto a livello di porte logiche, ovvero il più basso livello
	possibile. Questa è la simulazione più lenta in assoluto.
	\item \textbf{RTL (Register Transfer Level)}
	La rappresentazione in questo caso è vista come un flusso di informazioni che
	percorre il sistema, i cui risultati vengono salvati nei registri. Anche questa
	simulazione è lenta poiché molto vicina all'hardware.
	Essa può essere \textit{cycle accurate}, nella quale l'unità minima di elaborazione è il ciclo
	di clock dove è considerato come importante quello che avviene all'inizio e alla fine di esso,
	oppure può essere \textit{instruction accurate}, nella quale l'unità minima diventa la singola istruzione.
	\item \textbf{Behavioural}\\
	Questa rappresentazione descrive le funzionalità facendo una stima dei cicli di clock che impiega
	\item \textbf{Transactional}\\
	In questo caso non ho nemmeno il dettaglio della funzionalità, ma vado solamente a descrivere
	le interazioni tra i singoli moduli hardware. La simulazione in questa modalità è rapida.
\end{itemize}

Ogni livello di descrizione ha dei linguaggi più adatti di altri nonostante siano stati fatti diversi
sforzi di crearne uno adatto a tutti.
A livello RT, quelli più diffusi sono \textit{VHDL} e \textit{Verilog}.
\textit{SystemC}, nonostante sia in grado di scrivere più o meno bene a livello RT, mostra la sua
potenza espressiva agli altri livelli.

\section{Hardware description languages}
Gli \textit{Hardware Description Languages} (HDL) sono nati per risolvere una serie di problemi.
Prima di essi l'hardware veniva progettato a mano e senza alcuna procedura standardizzata. Questo
approccio però è prono ad errori e soprattutto incompatibile con le tempistiche richieste al giorno d'oggi.
Quando progetto del software mi basta pensare all'algoritmo astratto e codificarlo in un preciso linguaggio
di programmazione. Può anche succedere che il linguaggio che utilizzo sia multipiattaforma, per cui non mi dovrò
minimamente preoccupare di dove e come verrà eseguito, poiché so che a prescindere dall'architettura sul quale
verrà eseguito darà sempre lo stesso risultato.
Quando progettiamo hardware invece non possiamo permetterci questo lusso in quanto l'architettura semplicemente
non c'è, siamo noi a costruirla.

Prendiamo in esempio un programma scritto in linguaggio C e osserviamo le assunzioni che facciamo senza nemmeno pensare.\\
\begin{minted}[tabsize=2]{c}
#include <stdio.h>

int gcd(int xi, int yi){
   	int x, y, temp;
   	
   	x = xi;
   	y = yi;
   	while(x > 0){
   		if(x <= y){
   			temp = y;
   			y = x;
   			x = temp;
   		}
   		x = x - y;
   	}
   	return(y);
}

int main(int argc, char *argv[]){
   	int xi, yi, ou;
    
    scanf("%d %d", &xi, &yi);
    ou = gcd(xi, yi);
    printf("%d\n", ou);
    
    return 0;
}
\end{minted}
I requisiti hardware per poter eseguire questo programma sono:
\begin{itemize}
    \item \textbf{Input/Output}\\
    \textit{SW}: \texttt{printf}, \texttt{scanf}, ...\\
    \textit{HW}: interfacce di I/O
    \item \textbf{Temporizzazione}\\
    \textit{SW}: istruzioni vengono eseguite alla velocità del ciclo di clock\\
    \textit{HW}: devono essere definiti uno o più segnali di clock (e le istruzioni possono impiegare diversi cicli di clock)
    \item \textbf{Dimensioni variabili}\\
    \textit{SW}: dimensioni implicite sono nascoste\\
    \textit{HW}: devo tenere conto delle dimensioni, in quanto sto creando di fisico che poi corrisponderà alla variabile
    \item \textbf{Operazioni}\\
    \textit{SW}: esistono librerie per ogni tipo di operazioni\\
    \textit{HW}: difficili dato che devo fare un circuito apposito. Se poi vogliamo anche i numeri in virgola mobile, il circuito aumenta molto in complessità
    \item \textbf{Identificazione elementi di memoria}\\
    \textit{SW}: non guardo se una variabile va nei registri o nella RAM, la uso e basta\\
    \textit{HW}: devo sapere che spazio andrà a occupare, c'è una bella differenza tra registro e memoria
    \item \textbf{Sincronizzazione moduli}\\
    \textit{SW}: spesso lavoro in maniera sequenziale\\
    \textit{HW}: per com'è strutturato, l'hardware lavora tantissimo in parallelo        
\end{itemize}
Quando scriviamo l'algoritmo software facciamo quindi molte assunzioni, del tutto legittime, ma che non possiamo di certo dare per 
scontato quando invece progettiamo un modulo hardware.

\noindent
La prima cosa di cui bisogna preoccuparsi quando è definire le porte d'ingresso e di uscita. Una volta completata
l'identificazione delle porte, bisogna individuare la modalità con cui andremo a progettare il modulo,
che nel nostro caso sarà una FSMD, ovvero una macchina a stati finiti combinata con un datapath. 
Nella definizione della FSM e del DP, bisogna tenere conto degli eventuali vincoli: se ad esempio mi interessa un circuito veloce posso permettermi un DP più grande, mentre se voglio che il mio circuito occupi il minor spazio possibile dovrò allargare la FSM.
Riportiamo di seguito un possibile diagramma del modulo \texttt{gcd}.
\begin{figure}[h]
    \centering
    \begin{tikzpicture}
        \draw [->] (0,2) node[above]{XI} -- (1,2);
        \draw [->] (0,1) node[above]{YI} -- (1,1);
        \draw [<-] (2,3) -- (2,4) node[left]{CLOCK};
        \draw [<-] (4,3) -- (4,4) node[left]{RESET};
    
        \draw [fill=SkyBlue] (1,0) rectangle (5,3) node[pos=.5] {GCD};
        
        \draw [->] (5,1.5) -- (6,1.5) node [above] {OUT};
    \end{tikzpicture}
    \label{img:gcd_scheme}
    \caption{Schema di un modulo GCD}
\end{figure}
\\
Questa è una possibile implementazione in VHDL del modulo GCD, descritta dal punto di vista ingressi-uscite:
\begin{minted}[tabsize=2, escapeinside=??]{VHDL}
ENTITY gcd IS
    PORT (
        clock : IN bit; ?\tikzmark{gcd_input1}?
        reset : IN bit; ?\tikzmark{gcd_input2}?
        xi : IN unsigned (size-1 DOWNTO 0); ?\tikzmark{gcd_input3}?
        yi : IN unsigned (size-1 DOWNTO 0); ?\tikzmark{gcd_input4}?
        out : OUT unsigned (si?\tikzmark{gcd_varsize}?ze-1 DOWNTO 0) ?\tikzmark{gcd_output}?
    );
END gcd;
\end{minted}
\begin{tikzpicture}[remember picture]
    \draw[overlay, ->] ($(pic cs:gcd_input1)+(0,0.1)$) -- ($(pic cs:gcd_input1)+(5,-0.3)$) node [right]{porte di input};
    \draw[overlay, ->] ($(pic cs:gcd_input2)+(0,0.1)$) -- ($(pic cs:gcd_input1)+(5,-0.3)$);
    \draw[overlay, ->] ($(pic cs:gcd_input3)+(0,0.1)$) -- ($(pic cs:gcd_input1)+(5,-0.4)$);
    \draw[overlay, ->] ($(pic cs:gcd_input4)+(0,0.1)$) -- ($(pic cs:gcd_input1)+(5,-0.4)$);
    \draw[overlay, ->] ($(pic cs:gcd_output)+(0,0.1)$) -- ($(pic cs:gcd_output)+(1,0.1)$) node [right]{porta di output};
    \draw[overlay, ->] ($(pic cs:gcd_varsize)+(0,-0.05)$) -- ($(pic cs:gcd_varsize)+(0,-0.5)$) node [below]{dimensione variabile};
\end{tikzpicture}
mentre questa è la descrizione comportamentale del modulo hardware:
\begin{minted}[tabsize=2]{VHDL}
ARCHITECTURE behavioral OF gcd IS
BEGIN
    PROCESS
        VARIABLE x, y, temp : unsigned (size-1 DOWNTO 0);
    BEGIN
    WAIT UNTIL clock = '1'; 
    x := xi;
    y := yi;
    WHILE (x > 0) LOOP
        IF (x <= y) THEN
            temp := y;
            y := x;
            x := temp;
        END IF;
        x := x - y;
    END LOOP;
    ou <= y;
    END PROCESS;
END behavioral;
\end{minted}

\chapter{SystemC}
SystemC è un insieme di classi e macro del linguaggio C++ che forniscono un ambiente di simulazione \textit{event-driven}.
Queste classi permettono al progettista di simulare processi concorrenti,  che possono anche comunicare in un ambiente
real-time simulato, utilizzando segnali di qualsiasi tipo forniti da C++/SystemC o dall'utente.
Sebbene sia per certi versi simile a linguaggi come VHDL o Verilog, è più corretto definire SystemC un linguaggio di
modellazione di sistemi.
Lo standard è definito dalla \textit{Open SystemC Initiative} (OSCI), ora \textit{Accellera}, ed è stato
approvato dall'IEEE. Le caratteristiche salienti del SystemC sono:
\begin{itemize}
    \item \textbf{Concorrenza}\\
    Processi sincroni e asincroni
    \item \textbf{Comunicazione}\\
    IPC tramite segnali e canali
    \item \textbf{Nozione di tempo}\\
    Possibilità di avere cicli di clock multipli con fasi arbitrarie
    \item \textbf{Reattività}\\
    Possibilità di attesa su eventi
    \item \textbf{Tipi di dato hardware}\\
    Vettori di bit, interi a precisione arbitraria, ecc.
    \item \textbf{Simulazione}\\
    Kernel di simulazione incluso nella libreria
    \item \textbf{Debugging}\\
    Possibilità di utilizzare i debugger disponibili per C/C++ come GNU GDB
\end{itemize}
\begin{figure}
    \centering
    \begin{tikzpicture}        
        \node[draw, align=center](rtlevel) at (0,0) {SystemC Model\\RT Level};
        \node[draw, ellipse, align=center](refinement) at (0,1.2) {Refinement};
        \node[draw, ellipse, align=center](simulation) at (0,2.2) {Simulation};
        \node[draw, align=center](systemlevel) at (0,3.4) {SystemC Model\\System Level};
        
        \node[draw, align=center, fill=CornflowerBlue](fsmd) at (4,0) {FSMD Logic\\description};
        \node[draw, ellipse, align=center, fill=CornflowerBlue](synthesis) at (4,1.2) {Synthesis};
        \node[draw, align=center, fill=CornflowerBlue](vhdl) at (4,2.4) {VHDL/Verilog};
        
        \draw[->, thick] (systemlevel) -- (simulation);
        \draw[->, thick] (simulation) -- (refinement);
        \draw[->, thick] (refinement) -- (rtlevel);
        \draw[->, thick] (rtlevel.west) -- ++(-20pt,0pt) |- ($(simulation.west)+(0,-2pt)$);
        \draw[->, thick] ($(simulation.west)+(0,2pt)$) -- ++(-20pt,0pt) |- (systemlevel.west);
        
        \draw[->, thick] (vhdl) -- (synthesis);
        \draw[->, thick] (synthesis) -- (fsmd);
        \draw[->, thick] (rtlevel.south) -- ++(0pt,-10pt) -- ++(2,0) -- ++(0,4) node[midway] (middle){}  -- ++(2,0) node[above] {Automatic translation}-- (vhdl.north);
        \draw[->, thick, dashed] (middle) -- (synthesis.west);
    \end{tikzpicture}
    \label{img:systemc_design_flow}
    \caption{SystemC design flow}
\end{figure}    
Quando si progettano sistemi complessi, viene naturale dividere il progetto in sotto parti, che chiamiamo \textit{moduli},
ognuno dei quali svolge una specifica funzione e comunica con altri. In SystemC i moduli sono rappresentati nientemeno che 
da delle classi C++ e si specificano con la keyword \textit{\texttt{SC\_MODULE}}.
Un modulo contiene la definizione delle porte di input e di output, i segnali interni e la loro eventuale inizializzazione,
e i sottomoduli, che sono rappresentati nella loro forma minima dalle funzioni.
Inoltre ogni modulo contiene un metodo costruttore, identificato dalla macro \textit{\texttt{SC\_CTOR}}, che contiene
la dichiarazione di tutti i processi contenuti nel modulo e la sensitivity list associata a tali metodi, che specifica
i segnali ai quali ciascun metodo deve reagire.

I processi che vengono dichiarati all'interno di ogni modulo possono essere di tre tipi:
\begin{itemize}
    \item \textbf{Metodi}\\
    Sono dei processi che quando vengono attivati ogni volta che arriva uno dei segnali espressi nella sensitivity list.
    Si identificano con la keyword \textit{\texttt{SC\_METHOD}}.
    \item \textbf{Thread}\\
    Sono dei processi che possono essere attivati o sospesi, mediante la funzione \textbf{\texttt{wait()}}.
    A differenza dei metodi, possono essere eseguiti una volta sola durante la simulazione.
    Si identificano con la keyword \textit{\texttt{SC\_THREAD}}.
    \item \textbf{Clocked threads}\\
    Sono dei processi sensibili al segnale di clock. Sono stati dichiarati obsoleti.
\end{itemize}
La simulazione in SystemC è gestita dal kernel, che gestisce lo scheduling nel seguente modo:
\begin{itemize}[label={--}]
    \item Tutti i segnali di clock vengono aggiornati e tutti i processi sensibili ad esso vengono così attivati
    \item LOl
\end{itemize}

In SystemC è possibile descrivere sistemi hardware e software a diversi livelli di astrazione, permettendo quindi il co-design.
In particolare, si può descrivere a livello RT o TLM.
\section{SystemC RTL}
Nel \textit{Register Transfer Level} si può descrivere il funzionamento di un circuito digitale in termini di segnali,
registri e operazioni logiche.
\section{SystemC TLM}
\section{SystemC AMS}
\chapter{VHDL}

\end{document}
